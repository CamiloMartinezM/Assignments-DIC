\documentclass{article}

\usepackage{fancyhdr}
\usepackage{extramarks}
\usepackage{amsmath,siunitx}
\usepackage{amsthm}
\usepackage{bm}
\usepackage{amssymb}
\usepackage{amsfonts}
\usepackage{multirow}
\usepackage{tikz}
\usepackage[plain]{algorithm}
\usepackage{algpseudocode}
\usepackage{changepage}

\usetikzlibrary{automata,positioning}

% 
% Renewal of commands
%
\makeatletter
\renewenvironment{cases}[1][l]{\matrix@check\cases\env@cases{#1}}{\endarray\right.}
\def\env@cases#1{%
  \let\@ifnextchar\new@ifnextchar
  \left\lbrace\def\arraystretch{1.2}%
  \array{@{}#1@{\quad}l@{}}}
\makeatother

%
% Basic Document Settings
%

\topmargin=-0.45in
\evensidemargin=0in
\oddsidemargin=0in
\textwidth=6.5in
\textheight=9.0in
\headsep=0.25in

\linespread{1.1}

\pagestyle{fancy}
\lhead{\hmwkTeam}
\chead{\hmwkClass: \hmwkTitle}
\rhead{\firstxmark}
\lfoot{\lastxmark}
\cfoot{\thepage}

\renewcommand\headrulewidth{0.4pt}
\renewcommand\footrulewidth{0.4pt}

\setlength\parindent{0pt}

\newcommand{\setsep}{,    \ }

%
% Create Problem Sections
%

\newcommand{\enterProblemHeader}[1]{
    \nobreak\extramarks{}{Problem \hmwkNumber.\arabic{#1} continued on next page\ldots}\nobreak{}
    \nobreak\extramarks{Problem \hmwkNumber.\arabic{#1} (continued)}{Problem \hmwkNumber.\arabic{#1} continued on next page\ldots}\nobreak{}
}

\newcommand{\exitProblemHeader}[1]{
    \nobreak\extramarks{Problem \hmwkNumber.\arabic{#1} (continued)}{Problem \hmwkNumber.\arabic{#1} continued on next page\ldots}\nobreak{}
    \stepcounter{#1}
    \nobreak\extramarks{Problem \hmwkNumber.\arabic{#1}}{}\nobreak{}
}

\setcounter{secnumdepth}{0}
\newcounter{partCounter}
\newcounter{homeworkProblemCounter}
\setcounter{homeworkProblemCounter}{1}
\nobreak\extramarks{Problem \arabic{homeworkProblemCounter}}{}\nobreak{}

%
% Homework Problem Environment
%
% This environment takes an optional argument. When given, it will adjust the
% problem counter. This is useful for when the problems given for your
% assignment aren't sequential. See the last 3 problems of this template for an
% example.
%
\newenvironment{homeworkProblem}[2][-2]{
    \ifnum#1>0
        \setcounter{homeworkProblemCounter}{#1}
    \fi
    \section{Problem \hmwkNumber.\arabic{homeworkProblemCounter} #2}
    \setcounter{partCounter}{1}
    \enterProblemHeader{homeworkProblemCounter}
}{
    \exitProblemHeader{homeworkProblemCounter}
}

%
% Homework Details
%   - Title
%   - Due date
%   - Class
%   - Section/Time
%   - Instructor
%   - Author
%
\newcommand{\hmwkNumber}{H8}
\newcommand{\hmwkTitle}{Homework Assignment \hmwkNumber}
\newcommand{\hmwkClass}{DIC}
\newcommand{\hmwkTeam}{Team \#11}
\newcommand{\hmwkAuthorName}{\hmwkTeam: \\ Camilo Martínez 7057573, cama00005@stud.uni-saarland.de \\ Honglu Ma 7055053, homa00001@stud.uni-saarland.de \\ Maria Sibi 7058469, mama00047@stud.uni-saarland.de}

%
% Title Page
%

\title{
    % \vspace{2in}
    \textmd{\textbf{\hmwkClass:\ \hmwkTitle}}\\
}

\author{\hmwkAuthorName}
\date \today

\renewcommand{\part}[1]{\textbf{\large Part \Alph{partCounter}}\stepcounter{partCounter}\\}

%
% Various Helper Commands
%

% Useful for algorithms
\newcommand{\alg}[1]{\textsc{\bfseries \footnotesize #1}}

% For derivatives
\newcommand{\deriv}[1]{\frac{\mathrm{d}}{\mathrm{d}x} (#1)}

% For partial derivatives
\newcommand{\pderiv}[2]{\frac{\partial}{\partial #1} (#2)}

% Integral dx
\newcommand{\dx}{\mathrm{d}x}

% Define argmax and argmin
\DeclareMathOperator*{\argmax}{argmax}
\DeclareMathOperator*{\argmin}{argmin}

% Alias for the Solution section header
\newcommand{\solution}{\textbf{\large Solution}}

% Probability commands: Expectation, Variance, Covariance, Bias
\newcommand{\E}{\mathrm{E}}
\newcommand{\Var}{\mathrm{Var}}
\newcommand{\Cov}{\mathrm{Cov}}
\newcommand{\Bias}{\mathrm{Bias}}

\nocite{*}

\begin{document}

\maketitle
\begin{homeworkProblem}{Anisotropic Image-Driven Regularisation with Robustified Data Term}
For this energy functional, the data term is $\Psi(u^TJu)$ we have
\begin{align*}
    \partial_{u_1} \Psi(u^TJu)  &= \Psi'(u^TJu)\partial_{u_1} u^TJu\\
                                &= \Psi'(u^TJu)\partial_{u_1} \left[
                                    \begin{pmatrix}
                                        u_1 & u_2 & 1\\
                                    \end{pmatrix}
                                    \begin{pmatrix}
                                        J_{1,1} & J_{1,2} & J_{1,3}\\
                                        J_{2,1} & J_{2,2} & J_{2,3}\\
                                        J_{3,1} & J_{3,2} & J_{3,3}
                                    \end{pmatrix}
                                    \begin{pmatrix}
                                        u_1\\
                                        u_2\\
                                        1
                                    \end{pmatrix}
                                \right]\\
                                &= \Psi'(u^TJu)\partial_{u_1} \left[
                                    \begin{pmatrix}
                                        u_1J_{1,1} + u_2J_{2,1} + J_{3,1} & u_1J_{1,2} + u_2J_{2,2} + J_{3,2} & u_1J_{1,3} + u_2J_{2,3} + J_{3,3}\\
                                    \end{pmatrix}
                                    \begin{pmatrix}
                                        u_1\\
                                        u_2\\
                                        1
                                    \end{pmatrix}
                                    \right]\\
                                &= \Psi'(u^TJu)\partial_{u_1} (u_1^2J_{1,1} + u_1u_2(J_{1,2} + J_{2,1})+ u_1(J_{1,3} + J_{3,1}) + u_2^2J_{2,2} + u_2(J_{2,3}+J_{3,3}) + J_{3,3})\\
                                &= \Psi'(u^TJu)((J_{1,1} + J_{1,1})u_1 + (J_{1,2} + J_{2,1})u_2 + (J_{1,3} + J_{3,1}))\\
                                \intertext{As we know the motion tensor $J = K_\sigma\ast(\nabla_3f\nabla_3f^T)$ which is symmetry so we can simplify the equation as}
                                &= 2\Psi'(u^TJu)(J_{1,1}u_1 + J_{1,2}u_2 + J_{1,3})
\end{align*}
The smoothness term is provided in Lecture 15 Slide 17.
By plugging in the equation we got from the data term above we have
\[
    \partial_tu_1 = \mathrm{div}(D\nabla u_1) - \frac{1}{\alpha}\Psi'(u^TJu)(J_{1,1}u_1 + J_{1,2}u_2 + J_{1,3})
\]
The derivation is similar for $u_2$ except the result because we take the partial w.r.t. $u_2$ this time
\[
    \partial_{u_2} \Psi(u^TJu) = 2 \, \Psi'(u^TJu)(J_{1,2}u_1 + J_{2,2}u_2 + J_{2,3})
\]
and similarily we have
\[
    \partial_tu_2 = \mathrm{div}(D\nabla u_2) - \frac{1}{\alpha}\Psi'(u^TJu)(J_{1,2}u_1 + J_{2,2}u_2 + J_{2,3})
\]
\end{homeworkProblem}
\begin{homeworkProblem}{Flow-Driven Anisotropic Regularisation}
    Let $M = \sum_{i=1}^2\nabla u_i\nabla u_i^T$ we have the following Euler-Lagrange Equation
    \begin{align*}
        0   &= \sum_{i=1}^2\partial_{x_i}\partial{{u_1}_{x_i}}\mathrm{tr}(\Psi(M)) - \frac{1}{\alpha}\partial_{u_1}(f_{x_1}u_1 + f_{x_2}u_2 + f_{x_3})^2\\
            &= \sum_{i=1}^2\partial_{x_i}\partial{{u_1}_{x_i}}\mathrm{tr}(\Psi(M)) - \frac{2}{\alpha}f_{x_1}(f_{x_1}u_1 + f_{x_2}u_2 + f_{x_3})\\ 
            &= \sum_{i=1}^2\partial_{x_i}\mathrm{tr}(\partial{{u_1}_{x_i}}\Psi(M)) - \frac{2}{\alpha}f_{x_1}(f_{x_1}u_1 + f_{x_2}u_2 + f_{x_3})\\ 
            &= \sum_{i=1}^2\partial_{x_i}\mathrm{tr}(\Psi'(M)\partial{{u_1}_{x_i}}M) - \frac{2}{\alpha}f_{x_1}(f_{x_1}u_1 + f_{x_2}u_2 + f_{x_3})\\ 
            \intertext{Using Hint 1}
            &= \sum_{i=1}^2\partial_{x_i}\mathrm{tr}(\Psi'(M)(e_i\nabla u_1^T + \nabla u_1e_i^T)) - \frac{2}{\alpha}f_{x_1}(f_{x_1}u_1 + f_{x_2}u_2 + f_{x_3})\\ 
            \intertext{Let $\mu_j$ and $v_j$ denote the j-th eigenvalue and eigenvector of M, using Hint 4}
            &= \sum_{i=1}^2\partial_{x_i}\mathrm{tr}(\sum_{j}\Psi'(\mu_j)(v_jv_j^T)(e_i\nabla u_1^T + \nabla u_1e_i^T)) - \frac{2}{\alpha}f_{x_1}(f_{x_1}u_1 + f_{x_2}u_2 + f_{x_3})\\ 
            &= \sum_{i=1}^2\partial_{x_i}\mathrm{tr}(\sum_{j}\Psi'(\mu_j)(v_jv_j^Te_i\nabla u_1^T + v_jv_j^T\nabla u_1e_i^T)) - \frac{2}{\alpha}f_{x_1}(f_{x_1}u_1 + f_{x_2}u_2 + f_{x_3})\\ 
            \intertext{$v_j^Te_i$ and $v_j^T\nabla u_1$ are scalar we can put them in the front}
            &= \sum_{i=1}^2\partial_{x_i}\mathrm{tr}(\sum_{j}\Psi'(\mu_j)(v_j^Te_iv_j\nabla u_1^T + v_j^T\nabla u_1v_je_i^T)) - \frac{2}{\alpha}f_{x_1}(f_{x_1}u_1 + f_{x_2}u_2 + f_{x_3})\\   
            &= \sum_{i=1}^2\partial_{x_i}\mathrm{tr}(\sum_{j}\Psi'(\mu_j)(v_j^Te_iv_j\nabla u_1^T)) + \mathrm{tr}(\sum_{j}\Psi'(\mu_j)(v_j^T\nabla u_1v_je_i^T)) - \frac{2}{\alpha}f_{x_1}(f_{x_1}u_1 + f_{x_2}u_2 + f_{x_3})\\   
            \intertext{Using Hint 2}
            &= 2\sum_{i=1}^2\partial_{x_i}\sum_{j}\Psi'(\mu_j)(e_i^Tv_jv_j^T\nabla u_1) - \frac{2}{\alpha}f_{x_1}(f_{x_1}u_1 + f_{x_2}u_2 + f_{x_3})\\   
            &= 2\sum_{i=1}^2\partial_{x_i}e_i^T(\sum_{j}\Psi'(\mu_j)(v_jv_j^T))\nabla u_1 - \frac{2}{\alpha}f_{x_1}(f_{x_1}u_1 + f_{x_2}u_2 + f_{x_3})\\   
            \intertext{Using Hint 4 again}
            &= 2\sum_{i=1}^2\partial_{x_i}e_i^T\Psi'(M)\nabla u_1 - \frac{2}{\alpha}f_{x_1}(f_{x_1}u_1 + f_{x_2}u_2 + f_{x_3})\\   
            \intertext{Using Hint 2}
            &= 2 \, \mathrm{div}(\Psi'(M)\nabla u_1) - \frac{2}{\alpha}f_{x_1}(f_{x_1}u_1 + f_{x_2}u_2 + f_{x_3})\\   
            &= 2 \, \mathrm{div}(D\nabla u_1) - \frac{2}{\alpha}f_{x_1}(f_{x_1}u_1 + f_{x_2}u_2 + f_{x_3})\\   
            &= \mathrm{div}(D\nabla u_1) - \frac{1}{\alpha}f_{x_1}(f_{x_1}u_1 + f_{x_2}u_2 + f_{x_3})\\   
        \end{align*}
    So we have 
    \[
        \partial_tu_1 = \mathrm{div}(D\nabla u_1) - \frac{1}{\alpha}f_{x_1}(f_{x_1}u_1 + f_{x_2}u_2 + f_{x_3})
    \]
    The same derivation works for $u_2$ as well with 
    \[
        \partial_tu_2 = \mathrm{div}(D\nabla u_2) - \frac{1}{\alpha}f_{x_2}(f_{x_1}u_1 + f_{x_2}u_2 + f_{x_3})
    \]
\end{homeworkProblem}

\begin{homeworkProblem}{Design of Global Optic Flow Methods}

    According to Lecture 16, we can define the following data term \(M(D^1 H, \mathbf{u})\) given the cosine and sine of the hue channel \(H\) with an equal weight of $1/2$ for both:
    \begin{equation}\label{H8.2.1}
        M(D^1 H, \bm{u}) = \frac{1}{2} (\bm{u}^\top \nabla_3 \cos H)^2 + \frac{1}{2} (\bm{u}^\top \nabla_3 \sin H)^2
    \end{equation}
    
    Since the hue channel $H$ of the HSV Colour Space is invariant under global or local multiplicative changes and local additive changes. Moreover, since the value of the hue channel $H$ is an angle, problems arise at the transition $2\pi \leftrightarrow 0$. Therefore, we use the vector $(\cos{H}, \sin{H})^\top$ instead, i.e. two features $\cos{H}$ and $\sin{H}$. This implies that, for an RGB image, we calculate the hue channel $H$ using the corresponding conversion and we discard the information given by the saturation $S$ which denotes the radius (distance to next grey tone); and the value $V$ which is the height (brightness). We do this because those two channels do not have invariance under local additive changes. This is of course a trade-off, i.e losing information in exchange for invariance. \\

    On the other hand, we follow Lecture 15 to incorporate Isotropic Flow-Driven Regularisation, we define the following smoothness term:
    \begin{equation}\label{H8.2.2}
        S_{IF}(\nabla f, \nabla \bm{u}) = \Psi(\lvert \nabla u_1 \rvert^2 + \lvert \nabla u_2 \rvert^2)
    \end{equation}
    and we choose \( \Psi(s^2) \) to be the Charbonnier penaliser:
    \begin{equation}\label{H8.2.3}
        \Psi(s^2) = 2\lambda^2 \left( \sqrt{1 + \frac{s^2}{\lambda^2}} - 1 \right)
    \end{equation}
    
    Finally, combining \eqref{H8.2.1} and \eqref{H8.2.2}, the energy functional becomes:
    \begin{equation}
        E(\bm{u}) = \int_\Omega \left[ \frac{1}{2} (\bm{u}^\top \nabla_3 \cos{H})^2 + \frac{1}{2} (\bm{u}^\top \nabla_3 \sin{H})^2 + \alpha \Psi(\lvert \nabla u_1 \rvert^2 + \lvert \nabla u_2 \rvert^2) \right] \, d\bm{x} \, d\bm{y} \, dt
    \end{equation}

    where the domain of integration \(\Omega\) spans the spatial dimensions \(x, y\) and the temporal dimension \(t\) over which the image sequence is defined. \\

    Following Lecture 15, we know the Euler-Lagrange equations take the form:
    \begin{align*}
    \partial_t u_1 &= \partial_{x_1} S_{\partial x_1 u_1} + \partial_{x_2} S_{\partial x_2 u_1} - \frac{1}{\alpha} \partial_{u_1} M, \\
    \partial_t u_2 &= \partial_{x_1} S_{\partial x_1 u_2} + \partial_{x_2} S_{\partial x_2 u_2} - \frac{1}{\alpha} \partial_{u_2} M
    \end{align*}    

    From the same lecture, we learned that, for Isotropic Flow-Driven Regularisation, the terms involving the smoothness term $S$ become $\text{div} \left( \Psi'(\lvert \nabla u_1 \rvert^2 + \lvert \nabla u_2 \rvert^2) \nabla u_i \right)$. Thus, the Euler-Lagrange equations for the designed energy functional end up being:
    \begin{align*}
        \partial_t u_1 &= \text{div} \left( \Psi'(\lvert \nabla u_1 \rvert^2 + \lvert \nabla u_2 \rvert^2) \nabla u_1 \right) - \frac{1}{2 \alpha} \left(\cos{H}_{x_1}(\bm{u}^\top \nabla_3 \cos{H}) + \sin{H}_{x_1}(\bm{u}^\top \nabla_3 \sin{H})\right), \\
        \partial_t u_2 &= \text{div} \left( \Psi'(\lvert \nabla u_1 \rvert^2 + \lvert \nabla u_2 \rvert^2) \nabla u_2 \right) - \frac{1}{2 \alpha} \left(\cos{H}_{x_2}(\bm{u}^\top \nabla_3 \cos{H}) + \sin{H}_{x_2}(\bm{u}^\top \nabla_3 \sin{H})\right)
    \end{align*}  
        
    
\end{homeworkProblem}
\begin{homeworkProblem}{Motion Tensors}
    \subsection*{(a)}
    \begin{align*}
        M   &= \sum_{i=1}^{3}(u^T\nabla_3|\nabla f_i|^2)^2\\
            &= \sum_{i=1}^{3}(u^T\nabla_3|\nabla f_i|^2(\nabla_3|\nabla f_i|^2)^Tu)\\
            &= u^T\sum_{i=1}^{3}(\nabla_3|\nabla f_i|^2(\nabla_3|\nabla f_i|^2)^T)u\\
            &= u^T\sum_{i=1}^{3}(\nabla_3 ({f_i}_{x_1}^2 + {f_i}_{x_2}^2)(\nabla_3 ({f_i}_{x_1}^2 + {f_i}_{x_2}^2))^T)u\\
            \intertext{thus we have the motion tensor $\sum_{i=1}^{3}(\nabla_3 ({f_i}_{x_1}^2 + {f_i}_{x_2}^2)(\nabla_3 ({f_i}_{x_1}^2 + {f_i}_{x_2}^2))^T)$}
    \end{align*}
\end{homeworkProblem}
\end{document}
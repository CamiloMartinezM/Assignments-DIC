\documentclass{article}

\usepackage{fancyhdr}
\usepackage{extramarks}
\usepackage{amsmath}
\usepackage{amsthm}
\usepackage{bm}
\usepackage{amssymb}
\usepackage{amsfonts}
\usepackage{multirow}
\usepackage{tikz}
\usepackage[plain]{algorithm}
\usepackage{algpseudocode}

\usetikzlibrary{automata,positioning}


%
% Basic Document Settings
%

\topmargin=-0.45in
\evensidemargin=0in
\oddsidemargin=0in
\textwidth=6.5in
\textheight=9.0in
\headsep=0.25in

\linespread{1.1}

\pagestyle{fancy}
\lhead{\hmwkTeam}
\chead{\hmwkClass: \hmwkTitle}
\rhead{\firstxmark}
\lfoot{\lastxmark}
\cfoot{\thepage}

\renewcommand\headrulewidth{0.4pt}
\renewcommand\footrulewidth{0.4pt}

\setlength\parindent{0pt}

\newcommand{\setsep}{,    \ }

%
% Create Problem Sections
%

\newcommand{\enterProblemHeader}[1]{
    \nobreak\extramarks{}{Problem \hmwkNumber.\arabic{#1} continued on next page\ldots}\nobreak{}
    \nobreak\extramarks{Problem \hmwkNumber.\arabic{#1} (continued)}{Problem \hmwkNumber.\arabic{#1} continued on next page\ldots}\nobreak{}
}

\newcommand{\exitProblemHeader}[1]{
    \nobreak\extramarks{Problem \hmwkNumber.\arabic{#1} (continued)}{Problem \hmwkNumber.\arabic{#1} continued on next page\ldots}\nobreak{}
    \stepcounter{#1}
    \nobreak\extramarks{Problem \hmwkNumber.\arabic{#1}}{}\nobreak{}
}

\setcounter{secnumdepth}{0}
\newcounter{partCounter}
\newcounter{homeworkProblemCounter}
\setcounter{homeworkProblemCounter}{1}
\nobreak\extramarks{Problem \arabic{homeworkProblemCounter}}{}\nobreak{}

%
% Homework Problem Environment
%
% This environment takes an optional argument. When given, it will adjust the
% problem counter. This is useful for when the problems given for your
% assignment aren't sequential. See the last 3 problems of this template for an
% example.
%
\newenvironment{homeworkProblem}[2][-2]{
    \ifnum#1>0
        \setcounter{homeworkProblemCounter}{#1}
    \fi
    \section{Problem \hmwkNumber.\arabic{homeworkProblemCounter} #2}
    \setcounter{partCounter}{1}
    \enterProblemHeader{homeworkProblemCounter}
}{
    \exitProblemHeader{homeworkProblemCounter}
}

%
% Homework Details
%   - Title
%   - Due date
%   - Class
%   - Section/Time
%   - Instructor
%   - Author
%
\newcommand{\hmwkNumber}{H6}
\newcommand{\hmwkTitle}{Homework Assignment \hmwkNumber}
\newcommand{\hmwkClass}{DIC}
\newcommand{\hmwkTeam}{Team \#11}
\newcommand{\hmwkAuthorName}{\hmwkTeam: \\ Camilo Martínez 7057573, cama00005@stud.uni-saarland.de \\ Honglu Ma 7055053, homa00001@stud.uni-saarland.de \\ Maria Sibi 7058469, mama00047@stud.uni-saarland.de}

%
% Title Page
%

\title{
    % \vspace{2in}
    \textmd{\textbf{\hmwkClass:\ \hmwkTitle}}\\
}

\author{\hmwkAuthorName}
\date \today

\renewcommand{\part}[1]{\textbf{\large Part \Alph{partCounter}}\stepcounter{partCounter}\\}

%
% Various Helper Commands
%

% Useful for algorithms
\newcommand{\alg}[1]{\textsc{\bfseries \footnotesize #1}}

% For derivatives
\newcommand{\deriv}[1]{\frac{\mathrm{d}}{\mathrm{d}x} (#1)}

% For partial derivatives
\newcommand{\pderiv}[2]{\frac{\partial}{\partial #1} (#2)}

% Integral dx
\newcommand{\dx}{\mathrm{d}x}

% Define argmax and argmin
\DeclareMathOperator*{\argmax}{argmax}
\DeclareMathOperator*{\argmin}{argmin}

% Alias for the Solution section header
\newcommand{\solution}{\textbf{\large Solution}}

% Probability commands: Expectation, Variance, Covariance, Bias
\newcommand{\E}{\mathrm{E}}
\newcommand{\Var}{\mathrm{Var}}
\newcommand{\Cov}{\mathrm{Cov}}
\newcommand{\Bias}{\mathrm{Bias}}

\nocite{*}

\begin{document}

\maketitle

\begin{homeworkProblem}{(Stability of Diffusion-Reaction Discretisations)}

    Consider the diffusion-reaction equation
    \[
    \frac{\partial u}{\partial t} = \text{div} \left( g(|\nabla u|^2) \nabla u \right) - \frac{u - f}{\alpha} \tag*{(1)}\label{first}
    \]
    with \(\alpha > 0\).

    With discretisation with the \textit{modified explicit scheme}
    \[
    \frac{u^{k+1} - u^k}{\tau} = A^k(u^k) u^k - \frac{1}{\alpha}(u^{k+1} - f) \tag*{(2)}\label{second}
    \]
    and the \textit{fully explicit scheme}
    \[
    \frac{u^{k+1} - u^k}{\tau} = A^k(u^k) u^k - \frac{1}{\alpha}(u^k - f) \tag*{(3)}\label{third}
    \]

    \subsection*{(a)}

    \subsection*{(b)}

    \subsection*{(c)}

    We learned from Lecture 11 that the \textit{modified explicit scheme} has the same stability limit as an \textit{explicit diffusion scheme} without reaction term. And from Lecture 5, we know that the latter has an stability criterion for $\tau$ given by:
    
    \[
        \tau < \frac{1}{\max_i {\lvert a_{i,i}(\bar{u}(t)^k) \rvert}}  
    \]

    In this case, for $h_1 = h_2 = 1$, $0 < g(s^2) \leq 5$
    
    \[
        |a_{i,j}| = \sum_{n=1}^{2} \sum_{k \in N_n(i)} \frac{g_k + g_i}{2h_n^2} \leq \sum_{n=1}^{2} \sum_{k \in N_n(i)} \frac{5 + 5}{2} \leq \sum_{n=1}^{2} \sum_{k \in N_n(i)} 5
    \]

    In 2D, every pixel has at most four neighbours. This gives $\lvert a_{i,i} \rvert \leq 20$. Thus, the \textit{modified explicit scheme} must satisfy

    \[
        \tau < \frac{1}{20}   
    \]

    Note that this does not depend on $\alpha$.

    \subsection*{(d)}

    We can set up the \textit{fully explicit scheme} given by \ref{second} as follows: 

    \[
        u^{k+1} = \biggl(I + \tau A^k(u^k) - \frac{\tau}{\alpha} I\biggr) u^k + \frac{\tau}{\alpha} f
    \]

    Since $\tau > 0$ and $\alpha > 0$, $(\tau / \alpha) f$ is not of importance regarding stability. Thus, the stability of this equation is governed by the factor of $u^k$. From it, we get an inequality of the form:

    \[ 1 + \tau a_{i,i} - \frac{\tau}{\alpha} > 0 \]

    Where $a_{i,i}$ is a diagonal element of $A^k(u^k)$. We can rearrange the inequality to solve for \( \tau \), noting that the worst-case scenario would be when \( \lvert a_{i,i} \rvert \) is at its most. So, isolating $\tau$ on one side of the inequality gives us:

    \[ \tau \biggl(\max |a_{i,i}| - \frac{1}{\alpha}\biggr) < 1 \]

    Assuming \( \max |a_{i,i}| > \frac{1}{\alpha} \), we divide through by \( (\max |a_{i,i}| - \frac{1}{\alpha}) \):

    \[ \tau < \frac{1}{\max |a_{i,i}| - \frac{1}{\alpha}} \]

    If \( \max |a_{i,i}| \) is less than \( \frac{1}{\alpha} \), the explicit scheme could be unstable. \\
    Finally, as done in subsection (c), we know that in 2D, for $h_1 = h_2 = 1$, $0 < g(s^2) \leq 5$, we get $\lvert a_{i,i} \rvert \leq 20$. Thus, for $\alpha = 10$, the \textit{fully explicit scheme} must satisfy

    \[
        \tau < \frac{1}{20 - \frac{1}{10}} =  \frac{10}{199}
    \]

\end{homeworkProblem}

\begin{homeworkProblem}{(Primal-Dual Hybrid Gradient Algorithm)}

    Consider the equations of the primal-dual hybrid gradient (PDHG) algorithm for the primal variable \( u \) and the dual variable \( b \):
    
    \[
        b^{k+1} = \argmax_{b \in \mathbb{R}^{2N}} \left\{ -\iota_\alpha (b) + \langle b, D u^k \rangle - \frac{1}{2\tau} \| b - b^k \|^2 \right\} \tag*{(1)}\label{H6.2:first}
    \]
    \[
        u^{k+1} = \argmin_{u \in \mathbb{R}^N} \left\{ \frac{1}{2} \| u - f \|^2 + \langle D^T b^{k+1}, u \rangle + \frac{1}{2\sigma} \| u - u^k \|^2 \right\} \tag*{(2)}\label{H6.2:second}
    \]

    Let us define $P_{C_\alpha} ( \bar{b}^{k+1} )$ as follows:

    \[
        P_{C_\alpha} ( \bar{b}^{k+1} ) := \argmin_{b \in \mathbb{R}^{2N}} \left\{ \iota_\alpha (b) + \frac{1}{2} \| b - \bar{b}^{k+1} \|^2 \right\}
         = \argmin_{b \in C_\alpha} \left\{ \frac{1}{2} \| b - \bar{b}^{k+1} \|^2 \right\} 
    \]

    With that definition, consider the following derivation:
    \[
        \begin{split}
            P_{C_\alpha} (b^k + \tau D u^k)
            &= \argmin_{b \in C_\alpha} \left\{ \frac{1}{2} \| b - (b^k + \tau D u^k)\|^2 \right\} \\
            &= \argmin_{b \in C_\alpha} \left\{ \frac{1}{2} \left[\| b \|^2 - 2 \langle b, b^k + \tau D u^k \rangle + \| b^k + \tau D u^k \|^2 \right] \right\} \\
            &= \argmin_{b \in C_\alpha} \left\{ \frac{1}{2} \left[\| b \|^2 - 2 \langle b, b^k \rangle - 2 \langle b, \tau D u^k \rangle \right] \right\} \\
            &= \argmin_{b \in C_\alpha} \left\{ \frac{1}{2} \left[\| b \|^2 - 2 \langle b, b^k \rangle + \| b^k \|^2 - 2 \langle b, \tau D u^k \rangle \right] \right\} \\
            &= \argmin_{b \in C_\alpha} \left\{ \frac{1}{2} \left[\| b - b^k \|^2 - 2 \langle b, \tau D u^k \rangle \right] \right\} \\
            &= \argmin_{b \in C_\alpha} \left\{ \frac{1}{2} \| b - b^k \|^2 - \tau \langle b, D u^k \rangle \right\} \\
            &= \argmin_{b \in \mathbb{R}^{2N}} \left\{ \iota_\alpha (b) + \frac{1}{2} \| b - b^k \|^2 - \tau \langle b, D u^k \rangle \right\} \\
            &= \argmax_{b \in \mathbb{R}^{2N}} \left\{ - \frac{1}{\tau} \left[ \iota_\alpha (b) + \frac{1}{2} \| b - b^k \|^2 - \tau \langle b, D u^k \rangle \right] \right\} \\
            &= \argmax_{b \in \mathbb{R}^{2N}} \left\{ - \iota_\alpha (b) - \frac{1}{2\tau} \| b - b^k \|^2 + \langle b, D u^k \rangle \right\}
        \end{split}
    \]

    Which is the same expression given by \ref{H6.2:first}. \\ 
    On the other hand, consider \ref{H6.2:second}. The $\argmin_{u \in \mathbb{R}^N}$ operator implies taking the gradient with respect to $u$ and setting it to zero. Thus, 

    \[
        \begin{split}
            0 &= \nabla_u \left\{ \frac{1}{2} \| u - f \|^2 + \langle D^T b^{k+1}, u \rangle + \frac{1}{2\sigma} \| u - u^k \|^2 \right\} \\ 
            &= \| u - f \| + D^T b^{k+1} + \frac{1}{\sigma} \| u - u^k \| \\ 
            \rightarrow u \left( 1 + \frac{1}{\sigma}\right) &= f - D^T b^{k+1} + \frac{1}{\sigma} u^k
        \end{split}
    \]

    Where, in this resulting equation, $u$ can be seen as the next step after $u^k$, that is $u^{k+1}$. Therefore,

    \[
        u^{k+1} = \frac{1}{1 + \frac{1}{\sigma}} \left( f - D^T b^{k+1} + \frac{1}{\sigma} u^k \right) \tag*{(3)}\label{H6.2:third}
    \]

    Finally, we can note that \ref{H6.2:second} is composed of a sum of convex functions, where two of them are parabolas (which are convex) and the remaining one, the dot product, can been as a plane (also convex). Since the sum of convex functions is also convex, we can safely conclude that the expression \ref{H6.2:third} gives us a unique minimizer of equation \ref*{H6.2:second}.

\end{homeworkProblem}

\begin{homeworkProblem}{(Primal-Dual Methods for TV Regularisation)}

\subsection*{(a)-(c)}
See attached images.

\subsection*{(d)}

\end{homeworkProblem}

\end{document}
\documentclass{article}

\usepackage{fancyhdr}
\usepackage{extramarks}
\usepackage{amsmath}
\usepackage{amsthm}
\usepackage{bm}
\usepackage{amssymb}
\usepackage{amsfonts}
\usepackage{multirow}
\usepackage{tikz}
\usepackage[plain]{algorithm}
\usepackage{algpseudocode}

\usetikzlibrary{automata,positioning}


%
% Basic Document Settings
%

\topmargin=-0.45in
\evensidemargin=0in
\oddsidemargin=0in
\textwidth=6.5in
\textheight=9.0in
\headsep=0.25in

\linespread{1.1}

\pagestyle{fancy}
\lhead{\hmwkTeam}
\chead{\hmwkClass: \hmwkTitle}
\rhead{\firstxmark}
\lfoot{\lastxmark}
\cfoot{\thepage}

\renewcommand\headrulewidth{0.4pt}
\renewcommand\footrulewidth{0.4pt}

\setlength\parindent{0pt}

\newcommand{\setsep}{,    \ }

%
% Create Problem Sections
%

\newcommand{\enterProblemHeader}[1]{
    \nobreak\extramarks{}{Problem \hmwkNumber.\arabic{#1} continued on next page\ldots}\nobreak{}
    \nobreak\extramarks{Problem \hmwkNumber.\arabic{#1} (continued)}{Problem \hmwkNumber.\arabic{#1} continued on next page\ldots}\nobreak{}
}

\newcommand{\exitProblemHeader}[1]{
    \nobreak\extramarks{Problem \hmwkNumber.\arabic{#1} (continued)}{Problem \hmwkNumber.\arabic{#1} continued on next page\ldots}\nobreak{}
    \stepcounter{#1}
    \nobreak\extramarks{Problem \hmwkNumber.\arabic{#1}}{}\nobreak{}
}

\setcounter{secnumdepth}{0}
\newcounter{partCounter}
\newcounter{homeworkProblemCounter}
\setcounter{homeworkProblemCounter}{1}
\nobreak\extramarks{Problem \arabic{homeworkProblemCounter}}{}\nobreak{}

%
% Homework Problem Environment
%
% This environment takes an optional argument. When given, it will adjust the
% problem counter. This is useful for when the problems given for your
% assignment aren't sequential. See the last 3 problems of this template for an
% example.
%
\newenvironment{homeworkProblem}[2][-2]{
    \ifnum#1>0
        \setcounter{homeworkProblemCounter}{#1}
    \fi
    \section{Problem \hmwkNumber.\arabic{homeworkProblemCounter} #2}
    \setcounter{partCounter}{1}
    \enterProblemHeader{homeworkProblemCounter}
}{
    \exitProblemHeader{homeworkProblemCounter}
}

%
% Homework Details
%   - Title
%   - Due date
%   - Class
%   - Section/Time
%   - Instructor
%   - Author
%
\newcommand{\hmwkNumber}{H5}
\newcommand{\hmwkTitle}{Homework Assignment \hmwkNumber}
\newcommand{\hmwkClass}{DIC}
\newcommand{\hmwkTeam}{Team \#11}
\newcommand{\hmwkAuthorName}{\hmwkTeam: Camilo Martínez 7057573, cama00005@stud.uni-saarland.de \\ Honglu Ma 7055053, homa00001@stud.uni-saarland.de}

%
% Title Page
%

\title{
    % \vspace{2in}
    \textmd{\textbf{\hmwkClass:\ \hmwkTitle}}\\
}

\author{\hmwkAuthorName}
\date \today

\renewcommand{\part}[1]{\textbf{\large Part \Alph{partCounter}}\stepcounter{partCounter}\\}

%
% Various Helper Commands
%

% Useful for algorithms
\newcommand{\alg}[1]{\textsc{\bfseries \footnotesize #1}}

% For derivatives
\newcommand{\deriv}[1]{\frac{\mathrm{d}}{\mathrm{d}x} (#1)}

% For partial derivatives
\newcommand{\pderiv}[2]{\frac{\partial}{\partial #1} (#2)}

% Integral dx
\newcommand{\dx}{\mathrm{d}x}

% Alias for the Solution section header
\newcommand{\solution}{\textbf{\large Solution}}

% Probability commands: Expectation, Variance, Covariance, Bias
\newcommand{\E}{\mathrm{E}}
\newcommand{\Var}{\mathrm{Var}}
\newcommand{\Cov}{\mathrm{Cov}}
\newcommand{\Bias}{\mathrm{Bias}}

\nocite{*}

\begin{document}

\maketitle

\begin{homeworkProblem}{(Parameter Adaptation Under Rescaling)}

    \subsection*{(a)}

Since the area grows 4 times by size and the spatial grid size remains 1, T will have to be 4 times larger as well thus $T'=992$ and since the grey value range scaled down by a factor of 255, the contrast paramter $\lambda$ which usually represents a quantile of the histogram of image gradient magnitude will be scaled down by 255 as well: $\lambda' = 0.016$ and for the smoothing parameter $\sigma$ decreases by half to achieve the same convolution result: $\sigma' = 0.5$.

    \subsection*{(b)}

\end{homeworkProblem}

\begin{homeworkProblem}{(Continuous Variational Regularisation)}

    \subsection*{(a)}
    Given the following energy functional $E(u)$
    \begin{equation}\label{first}
        E(u) = \frac{1}{2}\int_\Omega {\biggl((u-f)^2 - 2\alpha\lambda^2\exp{\biggl(-\frac{\lvert\nabla u\rvert^2}{2\lambda^2}\biggr)}\biggr)d\bm{x}}
    \end{equation}
    We can define $G(u) = 2E(u)$, another energy functional so that we get a familiar equation. For trivial reasons, a minimizer for $G(u)$ is also a minimizer for $E(u)$, since we are only multiplying by a constant of $2$. Thus,
    \begin{equation}\label{second}
        G(u) = 2E(u) = \int_\Omega {\biggl((u-f)^2 - 2\alpha\lambda^2\exp{\biggl(-\frac{\lvert\nabla u\rvert^2}{2\lambda^2}\biggr)}\biggr)d\bm{x}}
    \end{equation}
    From \eqref{second}, we can derive 
    \begin{equation}\label{third}
        F_u = 2(u - f)
    \end{equation}
    \begin{equation}\label{fourth}
        F_{u_x} = 2\alpha u_x\exp{\biggl(-\frac{\lvert\nabla u\rvert^2}{2\lambda^2}\biggr)}
    \end{equation}
    \begin{equation}\label{fifth}
        F_{u_y} = 2\alpha u_y\exp{\biggl(-\frac{\lvert\nabla u\rvert^2}{2\lambda^2}\biggr)}
    \end{equation}
    Plugging \eqref{third}, \eqref{fourth} and \eqref{fifth} onto the Euler-Lagrange Equation, we get
    \begin{equation}\label{sixth}
        \begin{split}
            0 &= F_u - \frac{\partial}{\partial_x}F_{u_x} - \frac{\partial}{\partial_y}F_{u_y} \\ 
            &= (u - f) - \partial_x\biggl(2\alpha u_x\exp{\biggl(-\frac{\lvert\nabla u\rvert^2}{2\lambda^2}\biggr)}\biggr) - \partial_y\biggl(2\alpha u_y\exp{\biggl(-\frac{\lvert\nabla u\rvert^2}{2\lambda^2}\biggr)}\biggr) \\ 
            &= (u - f) - \alpha\biggl[\frac{\partial}{\partial x}\biggl(u_x\exp{\biggl(-\frac{\lvert\nabla u\rvert^2}{2\lambda^2}\biggr)}\biggr) + \frac{\partial}{\partial y}\biggl(u_y\exp{\biggl(-\frac{\lvert\nabla u\rvert^2}{2\lambda^2}\biggr)}\biggr)\biggr] \\ 
            &= (u - f) - \alpha \, \mathrm{div}\biggl(\exp{\biggl(-\frac{\lvert\nabla u\rvert^2}{2\lambda^2}\biggr)} \, \nabla u\biggr) 
        \end{split} 
    \end{equation}
    With the following boundary condition
    \begin{equation}\label{seven}
        \begin{split}
            0 &= n^\top \begin{pmatrix}
                    F_{u_x} \\ 
                    F_{u_y}
                \end{pmatrix} \\
              &= n^\top \, 2\alpha \exp{\biggl(-\frac{\lvert\nabla u\rvert^2}{2\lambda^2}\biggr)}
              \begin{pmatrix}
                u_x \\ 
                u_y
                \end{pmatrix} \\
              &= n^\top 2\alpha \exp{\biggl(-\frac{\lvert\nabla u\rvert^2}{2\lambda^2}\biggr)} \nabla u
        \end{split}
    \end{equation}
    Since $\alpha > 0$ and $\exp{(-\lvert\nabla u\rvert^2 / 2\lambda^2)} > 0$, we obtain the reflecting boundary condition 
    \begin{equation}\label{eigth}
        0 = n^\top \nabla u = \partial_n u
    \end{equation}
    Where $n$ denotes a normal vector to the image boundary $\partial\Omega$. For a rectangular domain $\Omega$, these boundary conditions imply that: $F_{u_x} = 0$ at the left and right boundaries, where $n = (1, 0)^T$ and $n = (-1, 0)^T$; and$F_{u_y} = 0$ at the top and bottom boundaries, where $n = (0, 1)^\top$ and $n = (0, -1)^\top$. \\

    From \eqref{sixth}, we can conclude that the Perona-Malik diffusion filter is approximated by this energy functional, where the diffusivity \(g(s^2)\) is given by the following familiar term
    \[
        g(s^2) = \exp{\biggl(-\frac{s^2}{2\lambda^2}\biggr)}
    \]
    
    \subsection*{(b)}

    The expression we got can be rewritten as 
    \[
        \frac{(u - f)}{\alpha} = \mathrm{div}\biggl(\exp{\biggl(-\frac{\lvert\nabla u\rvert^2}{2\lambda^2}\biggr)} \, \nabla u\biggr) 
    \]
    Which is very similar to the following expression that we studied in Lecture 4 (Nonlinear Isotropic Diffusion Filtering I)
    \[
        \partial_t u = \mathrm{div}(g(\lvert\nabla u\rvert^2) \nabla u)
    \] 
    \[
        \partial_n u = 0
    \]
    Where the diffusivity $g$ satisfies (C1) and (C3), smoothness and positivity. And we know the theoretical results of this model, mainly Well-Posedness and Regularity, in particular, the existence of a unique solution $u$. \\

    Moreover, we learned from the lecture that these kinds of variational methods do not involve a stopping time $T$, unlike the diffusion models we studied. For these, what we want to achieve is the steady-state solution (the unique minimiser) the fastest way possible, but the iterative process can go on without resulting in a final flat image. 

    \subsection*{(c)}
    In (a) we saw that the the Perona-Malik diffusion filter is approximated by this energy functional, and the diffusivity \(g\) would be given by
    \[
        g(\lvert\nabla u\rvert^2) = \exp{\biggl(-\frac{\lvert\nabla u\rvert^2}{2\lambda^2}\biggr)}
    \]
    Let us now consider the flux function 
    \[
        \varPhi(s) = s g(s) = s \exp{\biggl(-\frac{s^2}{2\lambda^2}\biggr)}
    \]
    And its derivative
    \[
        \begin{split}
            \varPhi'(s) &= \exp{\biggl(-\frac{s^2}{2\lambda^2}\biggr)} - \frac{s^2}{\lambda^2} \exp{\biggl(-\frac{s^2}{2\lambda^2}\biggr)} \\
            &= \exp{\biggl(-\frac{s^2}{2\lambda^2}\biggr)} - \frac{s^2}{\lambda^2} \exp{\biggl(-\frac{s^2}{2\lambda^2}\biggr)} \\
            &= \exp{\biggl(-\frac{s^2}{2\lambda^2}\biggr)} \biggl(1 - \frac{s^2}{\lambda^2}\biggr)
        \end{split}
    \]
    From there, it is apparent that the term $1 - s^2 / \lambda^2$ controls the sign of the flux. If $\lambda > s$, $\varPhi'(s) > 0$ thus having forward diffusion. And if $\lambda < s$, $\varPhi'(s) < 0$ thus having backward diffusion. It is clear that $\lambda$ itself separates both types of diffusion processes.

\end{homeworkProblem}

\begin{homeworkProblem}{(Energy Functional for Nonlinear Diffusion)}

From Lecture 10 Slide 22, we know that minimizing the 2-D energy function $$E(u) = \int_{a}^{b}((u-f)^2+\alpha\Psi(||\nabla u||^2)) \,dx $$ can be seen as an implicit time discretisation of the diffusion process $$\partial_tu = \mathrm{div}\biggl(\Psi'(||\nabla u||^2) \nabla u\biggr)$$ Here we have PDE given in form $$\partial_tu = \mathrm{div} \biggl(\frac{1}{1+\frac{||\nabla u||^2}{\lambda^2}}\nabla u\biggr)$$ so we have $$\Psi'(||\nabla u||^2) = \frac{1}{1+\frac{||\nabla u||^2}{\lambda^2}}$$ we solve the antiderivative of $\Psi'$ we get
$$\Psi(||\nabla u||^2) = \int\biggl(\frac{1}{1+\frac{||\nabla u||^2}{\lambda^2}}\biggr)\,d||\nabla u||^2 = \frac{1}{\lambda^2} \ln{|1+\lambda^2|}|\nabla u||^2| + C $$
replace this function of $\Psi$ in the energy functional above we will have the energy functional corresponding to the diffusion process.

\end{homeworkProblem}

\begin{homeworkProblem}{(Decorrelation Criterion)}
See attached code files.

\end{homeworkProblem}

\end{document}
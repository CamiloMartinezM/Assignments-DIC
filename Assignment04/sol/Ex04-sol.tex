\documentclass{article}

\usepackage{fancyhdr}
\usepackage{extramarks}
\usepackage{amsmath}
\usepackage{amsthm}
\usepackage{bm}
\usepackage{amssymb}
\usepackage{amsfonts}
\usepackage{tikz}
\usepackage[plain]{algorithm}
\usepackage{algpseudocode}

\usetikzlibrary{automata,positioning}


%
% Basic Document Settings
%

\topmargin=-0.45in
\evensidemargin=0in
\oddsidemargin=0in
\textwidth=6.5in
\textheight=9.0in
\headsep=0.25in

\linespread{1.1}

\pagestyle{fancy}
\lhead{\hmwkTeam}
\chead{\hmwkClass: \hmwkTitle}
\rhead{\firstxmark}
\lfoot{\lastxmark}
\cfoot{\thepage}

\renewcommand\headrulewidth{0.4pt}
\renewcommand\footrulewidth{0.4pt}

\setlength\parindent{0pt}

\newcommand{\setsep}{,    \ }

%
% Create Problem Sections
%

\newcommand{\enterProblemHeader}[1]{
    \nobreak\extramarks{}{Problem \hmwkNumber.\arabic{#1} continued on next page\ldots}\nobreak{}
    \nobreak\extramarks{Problem \hmwkNumber.\arabic{#1} (continued)}{Problem \hmwkNumber.\arabic{#1} continued on next page\ldots}\nobreak{}
}

\newcommand{\exitProblemHeader}[1]{
    \nobreak\extramarks{Problem \hmwkNumber.\arabic{#1} (continued)}{Problem \hmwkNumber.\arabic{#1} continued on next page\ldots}\nobreak{}
    \stepcounter{#1}
    \nobreak\extramarks{Problem \hmwkNumber.\arabic{#1}}{}\nobreak{}
}

\setcounter{secnumdepth}{0}
\newcounter{partCounter}
\newcounter{homeworkProblemCounter}
\setcounter{homeworkProblemCounter}{1}
\nobreak\extramarks{Problem \arabic{homeworkProblemCounter}}{}\nobreak{}

%
% Homework Problem Environment
%
% This environment takes an optional argument. When given, it will adjust the
% problem counter. This is useful for when the problems given for your
% assignment aren't sequential. See the last 3 problems of this template for an
% example.
%
\newenvironment{homeworkProblem}[2][-2]{
    \ifnum#1>0
        \setcounter{homeworkProblemCounter}{#1}
    \fi
    \section{Problem \hmwkNumber.\arabic{homeworkProblemCounter} #2}
    \setcounter{partCounter}{1}
    \enterProblemHeader{homeworkProblemCounter}
}{
    \exitProblemHeader{homeworkProblemCounter}
}

%
% Homework Details
%   - Title
%   - Due date
%   - Class
%   - Section/Time
%   - Instructor
%   - Author
%
\newcommand{\hmwkNumber}{H4}
\newcommand{\hmwkTitle}{Homework Assignment \hmwkNumber}
\newcommand{\hmwkClass}{DIC}
\newcommand{\hmwkTeam}{Team \#11}
\newcommand{\hmwkAuthorName}{\hmwkTeam: Camilo Martínez 7057573, Honglu Ma 7055053}

%
% Title Page
%

\title{
    % \vspace{2in}
    \textmd{\textbf{\hmwkClass:\ \hmwkTitle}}\\
}

\author{\hmwkAuthorName}
\date \today

\renewcommand{\part}[1]{\textbf{\large Part \Alph{partCounter}}\stepcounter{partCounter}\\}

%
% Various Helper Commands
%

% Useful for algorithms
\newcommand{\alg}[1]{\textsc{\bfseries \footnotesize #1}}

% For derivatives
\newcommand{\deriv}[1]{\frac{\mathrm{d}}{\mathrm{d}x} (#1)}

% For partial derivatives
\newcommand{\pderiv}[2]{\frac{\partial}{\partial #1} (#2)}

% Integral dx
\newcommand{\dx}{\mathrm{d}x}

% Alias for the Solution section header
\newcommand{\solution}{\textbf{\large Solution}}

% Probability commands: Expectation, Variance, Covariance, Bias
\newcommand{\E}{\mathrm{E}}
\newcommand{\Var}{\mathrm{Var}}
\newcommand{\Cov}{\mathrm{Cov}}
\newcommand{\Bias}{\mathrm{Bias}}


\begin{document}

\maketitle

\begin{homeworkProblem}{(Anisotropic Diffusion Modelling)}

\subsection*{(a)}
$$D(\nabla u_\sigma) :=  (v_1 \mid v_2)\,diag(g(\mu_1), 1)\,(v_1\mid v_2)^\top$$ where $v_1$ and $v_2$ are eigenvectors of $J_\rho(\nabla u_\sigma)$ and $\mu_1$ is the larger eigenvalue of $J_\rho(\nabla u_\sigma)$.
\subsection*{(b)}
When $\rho = 0$ we have $J_0(\nabla u_\sigma) = \nabla u_\sigma\,\nabla u_\sigma^\top$ from Classroom Work C4.1 we know that the two eigenvalues of $J_0(\nabla u_\sigma)$ are $\lambda_1 = 0$ and $\lambda_2 = ||\nabla u_\sigma||^2$ ; the two eigenvectors are $v_1 = \nabla u_\sigma^\perp$ and $v_2 = \nabla u_\sigma$. Since $\lambda_2 > \lambda_1$, $\mu_1 = \lambda_1$ 
With that, we construct our diffusion filter $D(\nabla u_\sigma) = \frac{\nabla u_\sigma^\perp}{||\nabla u_\sigma||}  \frac{(\nabla u_\sigma^\perp)^\top}{||\nabla u_\sigma||} + g(||\nabla u_\sigma||^2)\,\frac{\nabla u_\sigma}{||\nabla u_\sigma||}\,\frac{\nabla u_\sigma^\top}{||\nabla u_\sigma||}$.\\
\\
We can derive the diffusion filter: 
\begin{align}
&\partial_t u = div ((\frac{\nabla u_\sigma^\perp}{||\nabla u_\sigma||}  \frac{(\nabla u_\sigma^\perp)^\top}{||\nabla u_\sigma||} + g(||\nabla u_\sigma||^2)\,\frac{\nabla u_\sigma}{||\nabla u_\sigma||}\,\frac{\nabla u_\sigma^\top}{||\nabla u_\sigma||})\nabla u_\sigma) \\
& = div (\frac{\nabla u_\sigma^\perp}{||\nabla u_\sigma||}  \frac{(\nabla u_\sigma^\perp)^\top}{||\nabla u_\sigma||}\,\nabla u_\sigma +  g(||\nabla u_\sigma||^2)\,\frac{\nabla u_\sigma}{||\nabla u_\sigma||}\,\frac{\nabla u_\sigma^\top}{||\nabla u_\sigma||}\,\nabla u_\sigma) \\
& = div (0 +  g(||\nabla u_\sigma||^2)\,\frac{\nabla u_\sigma ||\nabla u_\sigma||^2}{||\nabla u_\sigma||^2})\\
& = div (g(||\nabla u_\sigma||^2)\,\nabla u_\sigma)
\end{align}\\
which is the same diffusion filter for non-linear isotropic diffusion.
\subsection*{(c)}
Given a small $\lambda$, $g(|\nabla u_\sigma|^2)$ will tend to $1$ which means both the direction along the edge and across will have diffusivity of $1$ and a large $\rho$ value will average the gradient directions which will give the image a 90 degree rotation along the original strip direction.

\end{homeworkProblem}

\begin{homeworkProblem}{(Directional Splitting of Anisotropic Diffusion)}

    First, we consider the right hand of the equation:
    \[
        \sum_{i = 0}^{3} {\partial_{\bm{e_i}} (w_i \partial_{\bm{e_i}} u)}
    \]
    Where we know that $\partial_{\bm{n}} u = \bm{n}^\mathsf{T} \nabla u$, the directional diffusivities $w_0$, $w_1$, $w_2$, $w_3$ are given by
    \[
        w_0 = a - \delta \setsep w_1 = \delta + b \setsep w_2 = c - \delta \setsep w_3 = \delta - b
    \]
    And the directions are given by
    \[
        \bm{e_0} = 
            \begin{pmatrix}
                1 \\ 
                0  
            \end{pmatrix}
        \setsep
        \bm{e_1} = \frac{1}{\sqrt{2}}
            \begin{pmatrix}
                1 \\ 
                1  
            \end{pmatrix}
        \setsep
        \bm{e_2} = 
            \begin{pmatrix}
                0 \\ 
                1  
            \end{pmatrix}
        \setsep
        \bm{e_3} = \frac{1}{\sqrt{2}} 
            \begin{pmatrix}
                -1 \\ 
                1  
            \end{pmatrix}
    \]
    
    For \(i = 0\), we have:
    \[
        {\partial_{\bm{e_0}} (w_0 \partial_{\bm{e_0}} u)} = 
            \begin{pmatrix}
                1 \\ 
                0  
            \end{pmatrix}^\mathsf{T} 
            \nabla \begin{bmatrix}
                (a - \delta)
                \begin{pmatrix}
                    1 \\ 
                    0  
                \end{pmatrix}^\mathsf{T} 
                \begin{pmatrix}
                    \partial_x u \\ 
                    \partial_y u  
                \end{pmatrix}
            \end{bmatrix}
        = \partial_x (a \partial_x u) - \partial_x (\delta \partial_x u)
    \]
    \\
    Similarly, for \(i = 2\), we get:
    \[
        {\partial_{\bm{e_2}} (w_2 \partial_{\bm{e_2}} u)} = 
            \begin{pmatrix}
                0 \\ 
                1  
            \end{pmatrix}^\mathsf{T} 
            \nabla \begin{bmatrix}
                (c - \delta)
                \begin{pmatrix}
                    0 \\ 
                    1  
                \end{pmatrix}^\mathsf{T} 
                \begin{pmatrix}
                    \partial_x u \\ 
                    \partial_y u  
                \end{pmatrix}
            \end{bmatrix}
        = \partial_y (c \partial_y u) - \partial_y (\delta \partial_y u)
    \]
    \\
    For \(i = 1\):
    \[
        \begin{split}
            {\partial_{\bm{e_1}} (w_1 \partial_{\bm{e_1}} u)} &= 
                \frac{1}{\sqrt{2}}
                \begin{pmatrix}
                    1 \\ 
                    1  
                \end{pmatrix}^\mathsf{T} 
                \nabla \begin{bmatrix}
                    \frac{1}{\sqrt{2}}
                    (\delta + b)
                    \begin{pmatrix}
                        1 \\ 
                        1  
                    \end{pmatrix}^\mathsf{T} 
                    \begin{pmatrix}
                        \partial_x u \\ 
                        \partial_y u  
                    \end{pmatrix}
                \end{bmatrix}
            \\
            &= \frac{1}{2} \partial_x (\delta \partial_x u) + \frac{1}{2} \partial_x (b \partial_x u) + \frac{1}{2} \partial_x (\delta \partial_y u) + \frac{1}{2} \partial_x (b \partial_y u) + 
            \frac{1}{2} \partial_y (\delta \partial_x u) + \frac{1}{2} \partial_y (b \partial_x u) +
            \frac{1}{2} \partial_y (\delta \partial_y u) + \frac{1}{2} \partial_y (b \partial_y u)
        \end{split}
    \]
    Finally, for \(i = 3\):
    \[
        \begin{split}
            {\partial_{\bm{e_3}} (w_3 \partial_{\bm{e_3}} u)} &= 
                \frac{1}{\sqrt{2}}
                \begin{pmatrix}
                    -1 \\ 
                    1  
                \end{pmatrix}^\mathsf{T} 
                \nabla \begin{bmatrix}
                    \frac{1}{\sqrt{2}}
                    (\delta - b)
                    \begin{pmatrix}
                        -1 \\ 
                        1  
                    \end{pmatrix}^\mathsf{T} 
                    \begin{pmatrix}
                        \partial_x u \\ 
                        \partial_y u  
                    \end{pmatrix}
                \end{bmatrix}
            \\
            &= \frac{1}{2} \partial_x (\delta \partial_x u) - \frac{1}{2} \partial_x (b \partial_x u) - \frac{1}{2} \partial_x (\delta \partial_y u) + \frac{1}{2} \partial_x (b \partial_y u) - 
            \frac{1}{2} \partial_y (\delta \partial_x u) + \frac{1}{2} \partial_y (b \partial_x u) +
            \frac{1}{2} \partial_y (\delta \partial_y u) - \frac{1}{2} \partial_y (b \partial_y u)
        \end{split}
    \]
    Summing up the terms for \(i = 1\) and \(i = 3\), we get:
    \[
        \begin{split}
            {\partial_{\bm{e_2}} (w_2 \partial_{\bm{e_2}} u)} + {\partial_{\bm{e_3}} (w_3 \partial_{\bm{e_3}} u)} 
            &=
            \partial_x (\delta \partial_x u) + \partial_x (b \partial_y u) + \partial_y (b \partial_x u) + \partial_y (\delta \partial_y u)
        \end{split}
    \]
    Then, summing up the resulting terms with the ones obtained for \(i = 0\) and \(i = 2\), we get:
    \begin{equation}\label{first}
        \begin{split}
            \sum_{i = 0}^{3} {\partial_{\bm{e_i}} (w_i \partial_{\bm{e_i}} u)} &= {\partial_{\bm{e_0}} (w_0 \partial_{\bm{e_0}} u)} + {\partial_{\bm{e_1}} (w_1 \partial_{\bm{e_1}} u)} + {\partial_{\bm{e_2}} (w_2 \partial_{\bm{e_2}} u)} + {\partial_{\bm{e_3}} (w_3 \partial_{\bm{e_3}} u)} 
            \\
            &=
            \partial_x (\delta \partial_x u) + \partial_x (b \partial_y u) + \partial_y (b \partial_x u) + \partial_y (\delta \partial_y u) \\
            &+ \partial_x (a \partial_x u) - \partial_x (\delta \partial_x u) + \partial_y (c \partial_y u) - \partial_y (\delta \partial_y u)
            \\
            &= \partial_x (a \partial_x u) + \partial_x (b \partial_y u) + \partial_y (b \partial_x u) + \partial_y (c \partial_y u)
        \end{split}
    \end{equation}
    On the other hand, let us consider the following derivation which uses the mathematical definition of the divergence of a vector:
    \begin{equation}\label{second}
        \begin{split}
            \mathbf{div}
            \begin{pmatrix}
                \begin{pmatrix}
                    a & b \\
                    b & c
                \end{pmatrix}
                \nabla u
            \end{pmatrix}
            &= \mathbf{div}
            \begin{pmatrix}
                \begin{pmatrix}
                    a & b \\
                    b & c
                \end{pmatrix}
                \begin{pmatrix}
                    \partial_x u \\ 
                    \partial_y u  
                \end{pmatrix}
            \end{pmatrix}
            \\
            &= \mathbf{div}
            \begin{pmatrix}
                a \partial_x u + b \partial_y u \\ 
                b \partial_x u + c \partial_y u  
            \end{pmatrix}
            \\
            &= 
            \partial_x (a \partial_x u) + \partial_x (b \partial_y u) + \partial_y (b \partial_x u) + \partial_y (c \partial_y u)
        \end{split}
    \end{equation}
    Comparing (\ref{first}) and (\ref{second}) term by term, we see that they are equal. Therefore,
    \[
        \sum_{i = 0}^{3} {\partial_{\bm{e_i}} (w_i \partial_{\bm{e_i}} u)} = \mathbf{div}
            \begin{pmatrix}
                \begin{pmatrix}
                    a & b \\
                    b & c
                \end{pmatrix}
                \nabla u
            \end{pmatrix}
    \]
\end{homeworkProblem}

\begin{homeworkProblem}{($\delta$-Stencil for Isotropic Diffusions)}
\subsection*{(a)}
The stencil for homogenous diffusion by setting $a = 1, b = 0, c = 1$ is:
\begin{center}
$\frac{1}{h^2}$
\begin{tabular}{ |c|c|c| } 
 \hline
 $\frac{\delta}{2}$ & $1 - \delta$ & $\frac{\delta}{2}$ \\
 \hline
 $1 - \delta$ & $-4 + 2\delta$ & $1 - \delta$ \\ 
 \hline
 $\frac{\delta}{2}$ & $1 - \delta$ & $\frac{\delta}{2}$ \\ 
\hline 
\end{tabular}
\end{center}
which is the same stencil in C2.2.
\subsection*{(b)}
From Problem H4.1b we get the diffusion tensor for non-linear isotropic diffusion: 
$$D(\nabla u_\sigma) = \frac{\nabla u_\sigma^\perp}{||\nabla u_\sigma||}  \frac{(\nabla u_\sigma^\perp)^\top}{||\nabla u_\sigma||} + g(||\nabla u_\sigma||^2)\,\frac{\nabla u_\sigma}{||\nabla u_\sigma||}\,\frac{\nabla u_\sigma^\top}{||\nabla u_\sigma||}$$\\
by expanding the gradient with its components, we get:
$a = \frac{u_y^2 + gu_x^2}{u_x^2+u_y^2}, b = \frac{(1-g)u_xu_y}{u_x^2+u_y^2}, c = \frac{u_x^2 + gu_y^2}{u_x^2+u_y^2}$ where $g = g(||u||^2)$. For simplification I omit $\sigma$. By plugging in the $a, b, c$ terms, we get the $\delta$-stencil. When $\delta = b$ it makes the top left weight 0.
\end{homeworkProblem}
\begin{homeworkProblem}{(Anisotropic Diffusion)}
\subsection*{(a)}
Naming for output files:\\
\textit{out1.pgm} corresponds to the output of part b $\sigma = 2, \rho = 0$\\
\textit{out2.pgm} corresponds to the output of part b $\sigma = 0, \rho = 2$ \\
\textit{out3.pgm} corresponds to the output of part c using standard discretisation \\
\textit{out4.pgm} corresponds to the output of part c using WWW \\
\subsection*{(b)}
When $\sigma = 0$ the noise of the original image is not removed thus some noise are treated as edges which give an unfulfilling result thus de-noise is important.
\subsection*{(c)}
The WWW stencil is prefered.
\end{homeworkProblem}
\end{document}
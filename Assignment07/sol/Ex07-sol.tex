\documentclass{article}

\usepackage{fancyhdr}
\usepackage{extramarks}
\usepackage{amsmath,siunitx}
\usepackage{amsthm}
\usepackage{bm}
\usepackage{amssymb}
\usepackage{amsfonts}
\usepackage{multirow}
\usepackage{tikz}
\usepackage[plain]{algorithm}
\usepackage{algpseudocode}
\usepackage{changepage}

\usetikzlibrary{automata,positioning}

% 
% Renewal of commands
%
\makeatletter
\renewenvironment{cases}[1][l]{\matrix@check\cases\env@cases{#1}}{\endarray\right.}
\def\env@cases#1{%
  \let\@ifnextchar\new@ifnextchar
  \left\lbrace\def\arraystretch{1.2}%
  \array{@{}#1@{\quad}l@{}}}
\makeatother

%
% Basic Document Settings
%

\topmargin=-0.45in
\evensidemargin=0in
\oddsidemargin=0in
\textwidth=6.5in
\textheight=9.0in
\headsep=0.25in

\linespread{1.1}

\pagestyle{fancy}
\lhead{\hmwkTeam}
\chead{\hmwkClass: \hmwkTitle}
\rhead{\firstxmark}
\lfoot{\lastxmark}
\cfoot{\thepage}

\renewcommand\headrulewidth{0.4pt}
\renewcommand\footrulewidth{0.4pt}

\setlength\parindent{0pt}

\newcommand{\setsep}{,    \ }

%
% Create Problem Sections
%

\newcommand{\enterProblemHeader}[1]{
    \nobreak\extramarks{}{Problem \hmwkNumber.\arabic{#1} continued on next page\ldots}\nobreak{}
    \nobreak\extramarks{Problem \hmwkNumber.\arabic{#1} (continued)}{Problem \hmwkNumber.\arabic{#1} continued on next page\ldots}\nobreak{}
}

\newcommand{\exitProblemHeader}[1]{
    \nobreak\extramarks{Problem \hmwkNumber.\arabic{#1} (continued)}{Problem \hmwkNumber.\arabic{#1} continued on next page\ldots}\nobreak{}
    \stepcounter{#1}
    \nobreak\extramarks{Problem \hmwkNumber.\arabic{#1}}{}\nobreak{}
}

\setcounter{secnumdepth}{0}
\newcounter{partCounter}
\newcounter{homeworkProblemCounter}
\setcounter{homeworkProblemCounter}{1}
\nobreak\extramarks{Problem \arabic{homeworkProblemCounter}}{}\nobreak{}

%
% Homework Problem Environment
%
% This environment takes an optional argument. When given, it will adjust the
% problem counter. This is useful for when the problems given for your
% assignment aren't sequential. See the last 3 problems of this template for an
% example.
%
\newenvironment{homeworkProblem}[2][-2]{
    \ifnum#1>0
        \setcounter{homeworkProblemCounter}{#1}
    \fi
    \section{Problem \hmwkNumber.\arabic{homeworkProblemCounter} #2}
    \setcounter{partCounter}{1}
    \enterProblemHeader{homeworkProblemCounter}
}{
    \exitProblemHeader{homeworkProblemCounter}
}

%
% Homework Details
%   - Title
%   - Due date
%   - Class
%   - Section/Time
%   - Instructor
%   - Author
%
\newcommand{\hmwkNumber}{H7}
\newcommand{\hmwkTitle}{Homework Assignment \hmwkNumber}
\newcommand{\hmwkClass}{DIC}
\newcommand{\hmwkTeam}{Team \#11}
\newcommand{\hmwkAuthorName}{\hmwkTeam: \\ Camilo Martínez 7057573, cama00005@stud.uni-saarland.de \\ Honglu Ma 7055053, homa00001@stud.uni-saarland.de \\ Maria Sibi 7058469, mama00047@stud.uni-saarland.de}

%
% Title Page
%

\title{
    % \vspace{2in}
    \textmd{\textbf{\hmwkClass:\ \hmwkTitle}}\\
}

\author{\hmwkAuthorName}
\date \today

\renewcommand{\part}[1]{\textbf{\large Part \Alph{partCounter}}\stepcounter{partCounter}\\}

%
% Various Helper Commands
%

% Useful for algorithms
\newcommand{\alg}[1]{\textsc{\bfseries \footnotesize #1}}

% For derivatives
\newcommand{\deriv}[1]{\frac{\mathrm{d}}{\mathrm{d}x} (#1)}

% For partial derivatives
\newcommand{\pderiv}[2]{\frac{\partial}{\partial #1} (#2)}

% Integral dx
\newcommand{\dx}{\mathrm{d}x}

% Define argmax and argmin
\DeclareMathOperator*{\argmax}{argmax}
\DeclareMathOperator*{\argmin}{argmin}

% Alias for the Solution section header
\newcommand{\solution}{\textbf{\large Solution}}

% Probability commands: Expectation, Variance, Covariance, Bias
\newcommand{\E}{\mathrm{E}}
\newcommand{\Var}{\mathrm{Var}}
\newcommand{\Cov}{\mathrm{Cov}}
\newcommand{\Bias}{\mathrm{Bias}}

\nocite{*}

\begin{document}

\maketitle

\begin{homeworkProblem}{(Multiple Choice)}
    \subsection*{(a)}
    Yes. Because time size $\tau$ has to satisfy $ \tau < \frac{1}{\underset{i}{|a_{i,i}(\bar{u}(t)^k)|}}$. One can observe that the larger the max of the diagnal element of $A$ is, the smaller the upperbound of $\tau$ is.
    \subsection*{(b)}
    Yes it does because a positive diffusivity will smooth the image.
    \subsection*{(c)}
    Yes, they are stable because neither of them the weights on the four corners of the stencils are 0.
    \subsection*{(d)}
    No because the diffusion process in this case is influenced by space.
    \subsection*{(e)}
    Yes because the regularisation parameter determine the one big time-step, i.e. how much the image diffuse.
    \subsection*{(f)}
    Yes it does.
\end{homeworkProblem}

\begin{homeworkProblem}{(Half-Quadratic Regularisation)}
 
    Given the following energy functional

    \begin{equation}\label{HQR(1)}
    E_{f}(u) = \int_{\Omega} \left( (u-f)^2 + \alpha \lambda^2 \ln{\left(1 + \frac{|\nabla u|^2}{\lambda^2}\right)} \right) dx, \quad \text{with } \alpha, \lambda > 0
    \end{equation}
    
    We can identify that

    \begin{equation}\label{HQR(2)}
    \Psi(\lvert \nabla u \rvert^2) = \lambda^2 \ln{\left(1 + \frac{|\nabla u|^2}{\lambda^2}\right)}
    \end{equation}

    And we know that \eqref{HQR(1)} has the following Euler-Lagrange equation:

    \begin{equation}\label{HQR(3)}
        0 = (u-f) - \alpha \, \text{div}(\Psi'(\lvert \nabla u \rvert^2) \nabla u)
    \end{equation}

    Now, consider the characteristic Euler-Lagrange equations of the Half-Quadratic Regularisation functional $E_{HQ}$:
    
    \begin{equation}\label{HQR(4)}
        0 = (u-f) - \alpha \, \text{div}(v \nabla u)
    \end{equation}
    \begin{equation}\label{HQR(5)}
        0 = \lvert \nabla u \rvert^2 + \eta'(v)
    \end{equation}

    Comparing \eqref{HQR(3)} and \eqref{HQR(4)}, we know that 

    \[
        v := \Psi'(\lvert \nabla u \rvert^2) = \lambda^2 \ln{\left(1 + \frac{|\nabla u|^2}{\lambda^2}\right)}
    \]

    From there, we can isolate the term $\lvert \nabla u \rvert^2$ and we get

    \[
        \lvert \nabla u \rvert^2 = \lambda^2 \left(\exp{\left(\frac{v}{\lambda^2}\right)} - 1\right)    
    \]

    Then, \eqref{HQR(5)} becomes 
    
    \[
        0 = \lvert \nabla u \rvert^2 + \lambda^2 \left(1 - \exp{\left(\frac{v}{\lambda^2}\right)}\right)
    \]

    This yields 
    
    \[
        \eta'(v) = \lambda^2 \left(1 - \exp{\left(\frac{v}{\lambda^2}\right)}\right) \rightarrow \eta'(v) = \lambda^2 \left(1 - \lambda^2 \exp{\left(\frac{v}{\lambda^2}\right)}\right)
    \]

    Finally, we get the following Half-Quadratic Regularisation formulation

    \[
        E_{HQ}(u,v) = \int_{\Omega} \left( (u-f)^2 + \alpha \left(v \lvert \nabla u \rvert^2 + \lambda^2 \left(1 - \lambda^2 \exp{\left(\frac{v}{\lambda^2}\right)}\right)\right)\right) dx  
    \]

\end{homeworkProblem}

\begin{homeworkProblem}{(Matrix-Valued Diffusion Filtering)}
    \subsection*{(a)}
    Yes, it remins positive semidefinite. From Assignment 4 we know that the eigenvalue of the diffusion tensor is $g(|\nabla u|^2)$ and $g(0)$ which are both positive.
    This result in a positive semidefinite matrix.
    \subsection*{(b)}
    \[
        \frac{u^{k+1}_{i,j} - u^k_{i,j}}{\alpha} = div(\Psi'(|\nabla u^k_{i,j}|^2)\nabla u^k_{i,j}), \forall i,j \in \{1, ..., m\}
    \]
\end{homeworkProblem}

\begin{homeworkProblem}{(Ambrosio–Tortorelli Approximation)}

    \subsection*{(a)} 
    \vspace*{-1.9em}
    \begin{adjustwidth}{2.5em}{0pt}

        The Gradient descent interpretation of the Ambrosio-Tortorelli Approximation is given by the following equations:

        \begin{equation}\label{AT(1)}
            \frac{\partial u}{\partial t} = \text{div}(v^2 \nabla u) - \beta (u-f)
        \end{equation}

        \begin{equation}\label{AT(2)}
            \frac{\partial v}{\partial t} = c\Delta v - \frac{v}{\alpha} |\nabla u|^2 + \frac{1-v}{4c}
        \end{equation}

        A simple modified explicit scheme for \eqref{AT(1)} can be written as follows:
        \[
            \begin{split}
                \frac{u_{i,j}^{k+1} - u_{i,j}^k}{\tau} &= \left[2 v_{i,j}^k \left[\frac{(v_{i+1,j}^k - v_{i-1,j}^k)(u_{i+1,j}^k - u_{i-1,j}^k)}{2h_x^2} + \frac{(v_{i,j+1}^k - v_{i,j-1}^k)(u_{i,j+1}^k - u_{i,j-1}^k)}{2h_y^2}\right] + (v_{i,j}^k)^2 \Delta u^k\right] - \dots \\ 
                &\quad \beta (u_{i,j}^{k+1} - f_{i,j}) \\
                \rightarrow u_{i,j}^{k+1} (1 + \beta \tau) &= \tau \left(2 v_{i,j}^k A^k + (v_{i,j}^k)^2 \Delta u^k + \beta f_{i,j}\right) + u_{i,j}^k \\
                \rightarrow u_{i,j}^{k+1} &= \frac{\tau}{1 + \beta \tau} \left(2 v_{i,j}^k A^k + (v_{i,j}^k)^2 \Delta u^k + \beta f_{i,j}\right) + \frac{1}{1 + \beta \tau} u_{i,j}^k
            \end{split}
        \]

        where $A^k$ and $\Delta u^k$ are defined at iteration $k$ as follows 

        \[
            A^k = \frac{(v_{i+1,j}^k - v_{i-1,j}^k)(u_{i+1,j}^k - u_{i-1,j}^k)}{2h_x^2} + \frac{(v_{i,j+1}^k - v_{i,j-1}^k)(u_{i,j+1}^k - u_{i,j-1}^k)}{2h_y^2}
        \]
        \[
            \Delta u^k = \frac{u_{i+1,j}^{k} - 2u_{i,j}^{k} + u_{i-1,j}^{k}}{h_x^2} + \frac{u_{i,j+1}^{k} - 2u_{i,j}^{k} + u_{i,j-1}^{k}}{h_y^2}
        \]

        For \eqref{AT(2)}, we can use an explicit approach:

        \[
            \begin{split}
                \frac{v_{i,j}^{k+1} - v_{i,j}^k}{\tau} &= c\Delta v_{i,j}^k - \frac{v_{i,j}^{k}}{\alpha} \left|\nabla u_{i,j}^{k}\right|^2 + \frac{1-v_{i,j}^k}{4c} \\
                v_{i,j}^{k+1} &= \tau \left[ c\Delta v_{i,j}^k - \frac{v_{i,j}^{k}}{\alpha} \left|\nabla u_{i,j}^{k}\right|^2 + \frac{1-v_{i,j}^k}{4c} \right] + v_{i,j}^k
            \end{split}
        \]

        where 
        
        \[
            \Delta v_{i,j}^k = \frac{v_{i+1,j}^{k} - 2v_{i,j}^{k} + v_{i-1,j}^{k}}{h_x^2} + \frac{v_{i,j+1}^{k} - 2v_{i,j}^{k} + v_{i,j-1}^{k}}{h_y^2}
        \]
        \[
            \lvert \nabla u_{i,j}^k \rvert^2 = \frac{(u_{i+1,j}^k - u_{i-1,j}^k)^2}{4h_x^2} + \frac{(u_{i+1,j}^k - u_{i-1,j}^k)^2}{4h_y^2}
        \]

    \end{adjustwidth}

\end{homeworkProblem}

\end{document}
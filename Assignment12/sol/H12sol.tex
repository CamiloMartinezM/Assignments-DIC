\documentclass{article}

\usepackage{fancyhdr}
\usepackage{extramarks}
\usepackage{amsmath,siunitx}
\usepackage{amsthm}
\usepackage{bm}
\usepackage{amssymb}
\usepackage{amsfonts}
\usepackage{multirow}
\usepackage{tikz}
\usepackage[plain]{algorithm}
\usepackage{algpseudocode}
\usepackage{changepage}

\usetikzlibrary{automata,positioning}

% 
% Renewal of commands
%
\makeatletter
\renewenvironment{cases}[1][l]{\matrix@check\cases\env@cases{#1}}{\endarray\right.}
\def\env@cases#1{%
  \let\@ifnextchar\new@ifnextchar
  \left\lbrace\def\arraystretch{1.2}%
  \array{@{}#1@{\quad}l@{}}}
\makeatother

%
% Basic Document Settings
%

\topmargin=-0.45in
\evensidemargin=0in
\oddsidemargin=0in
\textwidth=6.5in
\textheight=9.0in
\headsep=0.25in

\linespread{1.1}

\pagestyle{fancy}
\lhead{\hmwkTeam}
\chead{\hmwkClass: \hmwkTitle}
\rhead{\firstxmark}
\lfoot{\lastxmark}
\cfoot{\thepage}

\renewcommand\headrulewidth{0.4pt}
\renewcommand\footrulewidth{0.4pt}

\setlength\parindent{0pt}

\newcommand{\setsep}{,    \ }

%
% Create Problem Sections
%

\newcommand{\enterProblemHeader}[1]{
    \nobreak\extramarks{}{Problem \hmwkNumber.\arabic{#1} continued on next page\ldots}\nobreak{}
    \nobreak\extramarks{Problem \hmwkNumber.\arabic{#1} (continued)}{Problem \hmwkNumber.\arabic{#1} continued on next page\ldots}\nobreak{}
}

\newcommand{\exitProblemHeader}[1]{
    \nobreak\extramarks{Problem \hmwkNumber.\arabic{#1} (continued)}{Problem \hmwkNumber.\arabic{#1} continued on next page\ldots}\nobreak{}
    \stepcounter{#1}
    \nobreak\extramarks{Problem \hmwkNumber.\arabic{#1}}{}\nobreak{}
}

\setcounter{secnumdepth}{0}
\newcounter{partCounter}
\newcounter{homeworkProblemCounter}
\setcounter{homeworkProblemCounter}{1}
\nobreak\extramarks{Problem \arabic{homeworkProblemCounter}}{}\nobreak{}

%
% Homework Problem Environment
%
% This environment takes an optional argument. When given, it will adjust the
% problem counter. This is useful for when the problems given for your
% assignment aren't sequential. See the last 3 problems of this template for an
% example.
%
\newenvironment{homeworkProblem}[2][-2]{
    \ifnum#1>0
        \setcounter{homeworkProblemCounter}{#1}
    \fi
    \section{Problem \hmwkNumber.\arabic{homeworkProblemCounter} #2}
    \setcounter{partCounter}{1}
    \enterProblemHeader{homeworkProblemCounter}
}{
    \exitProblemHeader{homeworkProblemCounter}
}

%
% Homework Details
%   - Title
%   - Due date
%   - Class
%   - Section/Time
%   - Instructor
%   - Author
%
\newcommand{\hmwkNumber}{H12}
\newcommand{\hmwkTitle}{Homework Assignment \hmwkNumber}
\newcommand{\hmwkClass}{DIC}
\newcommand{\hmwkTeam}{Team \#11}
\newcommand{\hmwkAuthorName}{\hmwkTeam: \\ Camilo Martínez 7057573, cama00005@stud.uni-saarland.de \\ Honglu Ma 7055053, homa00001@stud.uni-saarland.de \\ Maria Sibi 7058469, mama00047@stud.uni-saarland.de}

%
% Title Page
%

\title{
    % \vspace{2in}
    \textmd{\textbf{\hmwkClass:\ \hmwkTitle}}\\
}

\author{\hmwkAuthorName}
\date \today

\renewcommand{\part}[1]{\textbf{\large Part \Alph{partCounter}}\stepcounter{partCounter}\\}

%
% Various Helper Commands
%

% Useful for algorithms
\newcommand{\alg}[1]{\textsc{\bfseries \footnotesize #1}}

% For derivatives
\newcommand{\deriv}[1]{\frac{\mathrm{d}}{\mathrm{d}x} (#1)}

% For partial derivatives
\newcommand{\pderiv}[2]{\frac{\partial}{\partial #1} (#2)}

% Integral dx
\newcommand{\dx}{\mathrm{d}x}

% Define argmax and argmin
\DeclareMathOperator*{\argmax}{argmax}
\DeclareMathOperator*{\argmin}{argmin}

% Alias for the Solution section header
\newcommand{\solution}{\textbf{\large Solution}}

% Probability commands: Expectation, Variance, Covariance, Bias
\newcommand{\E}{\mathrm{E}}
\newcommand{\Var}{\mathrm{Var}}
\newcommand{\Cov}{\mathrm{Cov}}
\newcommand{\Bias}{\mathrm{Bias}}

\begin{document}

\maketitle

\begin{homeworkProblem}{Curvature-Based Morphology}
    We have $\xi = \frac{1}{|\nabla u|}(-u_y, u_x)^\top$, 
    plug it in the formula we got in C12.1a:
    \begin{align*}
        \partial_{\xi\xi}u  &=  \xi^\top \mathrm{Hess}(u) \xi\\
                            &=  \frac{1}{|\nabla u|}(-u_y, u_x)\begin{pmatrix}
                                u_{xx}  &   u_{xy}\\
                                u_{yx}  &   u_{yy}
                            \end{pmatrix}\frac{1}{|\nabla u|}(-u_y, u_x)^\top\\
                            &=  \frac{1}{|\nabla u|^2}(-u_yu_{xx}+u_xu_{yx}, -u_yu_{xy}+u_xu_{yy})(-u_y, u_x)^\top\\
                            &=  \frac{1}{|\nabla u|^2}(u_y^2u_{xx}-u_yu_xu_{yx} + -u_xu_yu_{xy}+u_x^2u_{yy})\\
                            &=  \frac{u_y^2u_{xx} - 2u_xu_yu_{xy} + u_x^2u_{yy}}{u_x^2 + u_y^2}\\
    \end{align*}
    and from C12.1b we know that laplacian of u can be expressed by sum of second derivative of u in any two orthogonal normal vector direction,
    in the lecture we have $\Delta u = \partial_{\xi\xi}u + \partial_{\eta\eta}u$ where $\eta = \frac{1}{|\nabla u|}\nabla u$. We get:
    \begin{align*}
        \partial_{\xi\xi}u  &=  \Delta u - \partial_{\eta\eta}u\\
                            &=  \Delta u - \eta^\top\mathrm{Hess}(u)\eta\\
                            &=  \Delta u - \frac{1}{|\nabla u|}\nabla u^\top\mathrm{Hess}(u)\frac{1}{|\nabla u|}\nabla u\\
                            &=  \Delta u - \frac{1}{|\nabla u|^2}\nabla u^\top\mathrm{Hess}(u)\nabla u\\
    \end{align*}
    and for the third term we have:
    \begin{align*}
        |\nabla u|\mathrm{div}(\frac{\nabla u}{|\nabla u|}) &=  |\nabla u|(\partial_x\frac{u_x}{|\nabla u|} + \partial_y\frac{u_y}{|\nabla u|})\\
                                                            &=  |\nabla u|(\frac{|\nabla u|u_{xx} - u_x\partial_x|\nabla u|}{|\nabla u|^2} + \frac{|\nabla u|u_{yy} - u_y\partial_y|\nabla u|}{|\nabla u|^2})\\
                                                            &=  |\nabla u|(\frac{|\nabla u|u_{xx} - u_x\frac{u_xu_{xx} + u_yu_{yx}}{|\nabla u|} + |\nabla u|u_{yy} - u_y\frac{u_yu_{yy} + u_xu_{xy}}{|\nabla u|}}{|\nabla u|^2})\\
                                                            &=  \frac{|\nabla u|^2u_{xx} - u_x(u_xu_{xx} + u_yu_{yx}) + |\nabla u|^2u_{yy} - u_y(u_yu_{yy} + u_xu_{xy})}{|\nabla u|^2}\\
                                                            &=  \frac{(u_x^2+u_y^2)u_{xx} - u_x^2u_{xx} - u_xu_yu_{yx} + (u_x^2+u_y^2)u_{yy} - u_y^2u_{yy} - u_xu_yu_{xy}}{|\nabla u|^2}\\
                                                            &=  \frac{u_x^2u_{xx}+u_y^2u_{xx} - u_x^2u_{xx} - u_xu_yu_{yx} + u_x^2u_{yy}+u_y^2u_{yy} - u_y^2u_{yy} - u_xu_yu_{xy}}{|\nabla u|^2}\\
                                                            &=  \frac{u_y^2u_{xx} - 2u_xu_yu_{yx} + u_x^2u_{yy}}{u_x^2+u_y^2}\\
    \end{align*}
\end{homeworkProblem}

\begin{homeworkProblem}{Affine Invariant Arc-length}
    Consider matrix $A$ and vector $b$:
    \[
        A := \begin{pmatrix}
            a_{11} & a_{12}\\
            a_{21} & a_{22}\\
        \end{pmatrix}
        \qquad
        b := \begin{pmatrix}
            b_1\\
            b_2
        \end{pmatrix}
    \]
    such that $\mathrm{det}A = 1$.\\

    After affine transform ($c' = Ac + b$) we have 
    \[
        c'(p) = \begin{pmatrix}
            a_{11}x(p) + a_{12}y(p) + b_1\\
            a_{21}x(p) + a_{22}y(p) + b_2
        \end{pmatrix}
    \]
    We calculate the derivatives of $c$ after the transformation, i.e. $c'_p(p)$ and $c'_{pp}(p)$
    \[
        c'_p(p) = \begin{pmatrix}
            a_{11}x_p(p) + a_{12}y_p(p)\\
            a_{21}x_p(p) + a_{22}y_p(p)
        \end{pmatrix}
    \]
    One can observe that $c'_p(p) = Ac_p(p)$.
    Similarily we have $c'_{pp}(p) = Ac_{pp}(p)$.\\

    Now the new arc length function can be written as:
    \begin{align*}
        s'(p)   &= \int_{0}^{p} (\mathrm{det}(c'_p(p),c'_{pp}(p)))^\frac{1}{3}\,dp\\
                &= \int_{0}^{p} (\mathrm{det}(Ac_p(p),Ac_{pp}(p)))^\frac{1}{3}\,dp\\
                \intertext{Considering $(c_p(p),c_{pp}(p))$ is a matrix, $(Ac_p(p),Ac_{pp}(p))$ is the same as $A(c_p(p),c_{pp}(p))$}\\
                &= \int_{0}^{p} (\mathrm{det}(A(c_p(p),c_{pp}(p))))^\frac{1}{3}\,dp\\
                &= \int_{0}^{p} (\mathrm{det}(A)\mathrm{det}(c_p(p),c_{pp}(p)))^\frac{1}{3}\,dp\\
                &= \int_{0}^{p} (1\cdot\mathrm{det}(c_p(p),c_{pp}(p)))^\frac{1}{3}\,dp\\
                &= \int_{0}^{p} (\mathrm{det}(c_p(p),c_{pp}(p)))^\frac{1}{3}\,dp\\
    \end{align*}
    which is the same as the one before the transformation
\end{homeworkProblem}
\begin{homeworkProblem}{Mean Curvature Motion}
    \subsection*{(c)}
    Because under MCM shapes will become convex and vanish in the end.
    The fingerprint lines' detail is simplified and vanishes.
    \subsection*{(d)}
    The two shapes(squares) in \textbf{check1.pgm} stays connected after the evolution
    while they disconnect after using MCM on \textbf{check2.pgm}.
    This does not contradict to the theory that topologically connected objects remain connected under MCM (Lecture 23, Slide 10)
\end{homeworkProblem}

\end{document}

\documentclass{article}

\usepackage{fancyhdr}
\usepackage{extramarks}
\usepackage{amsmath,siunitx}
\usepackage{amsthm}
\usepackage{bm}
\usepackage{amssymb}
\usepackage{amsfonts}
\usepackage{multirow}
\usepackage{tikz}
\usepackage[plain]{algorithm}
\usepackage{algpseudocode}
\usepackage{changepage}

\usetikzlibrary{automata,positioning}

% 
% Renewal of commands
%
\makeatletter
\renewenvironment{cases}[1][l]{\matrix@check\cases\env@cases{#1}}{\endarray\right.}
\def\env@cases#1{%
  \let\@ifnextchar\new@ifnextchar
  \left\lbrace\def\arraystretch{1.2}%
  \array{@{}#1@{\quad}l@{}}}
\makeatother

%
% Basic Document Settings
%

\topmargin=-0.45in
\evensidemargin=0in
\oddsidemargin=0in
\textwidth=6.5in
\textheight=9.0in
\headsep=0.25in

\linespread{1.1}

\pagestyle{fancy}
\lhead{\hmwkTeam}
\chead{\hmwkClass: \hmwkTitle}
\rhead{\firstxmark}
\lfoot{\lastxmark}
\cfoot{\thepage}

\renewcommand\headrulewidth{0.4pt}
\renewcommand\footrulewidth{0.4pt}

\setlength\parindent{0pt}

\newcommand{\setsep}{,    \ }

%
% Create Problem Sections
%

\newcommand{\enterProblemHeader}[1]{
    \nobreak\extramarks{}{Problem \hmwkNumber.\arabic{#1} continued on next page\ldots}\nobreak{}
    \nobreak\extramarks{Problem \hmwkNumber.\arabic{#1} (continued)}{Problem \hmwkNumber.\arabic{#1} continued on next page\ldots}\nobreak{}
}

\newcommand{\exitProblemHeader}[1]{
    \nobreak\extramarks{Problem \hmwkNumber.\arabic{#1} (continued)}{Problem \hmwkNumber.\arabic{#1} continued on next page\ldots}\nobreak{}
    \stepcounter{#1}
    \nobreak\extramarks{Problem \hmwkNumber.\arabic{#1}}{}\nobreak{}
}

\setcounter{secnumdepth}{0}
\newcounter{partCounter}
\newcounter{homeworkProblemCounter}
\setcounter{homeworkProblemCounter}{1}
\nobreak\extramarks{Problem \arabic{homeworkProblemCounter}}{}\nobreak{}

%
% Homework Problem Environment
%
% This environment takes an optional argument. When given, it will adjust the
% problem counter. This is useful for when the problems given for your
% assignment aren't sequential. See the last 3 problems of this template for an
% example.
%
\newenvironment{homeworkProblem}[2][-2]{
    \ifnum#1>0
        \setcounter{homeworkProblemCounter}{#1}
    \fi
    \section{Problem \hmwkNumber.\arabic{homeworkProblemCounter} #2}
    \setcounter{partCounter}{1}
    \enterProblemHeader{homeworkProblemCounter}
}{
    \exitProblemHeader{homeworkProblemCounter}
}

%
% Homework Details
%   - Title
%   - Due date
%   - Class
%   - Section/Time
%   - Instructor
%   - Author
%
\newcommand{\hmwkNumber}{H13}
\newcommand{\hmwkTitle}{Homework Assignment \hmwkNumber}
\newcommand{\hmwkClass}{DIC}
\newcommand{\hmwkTeam}{Team \#11}
\newcommand{\hmwkAuthorName}{\hmwkTeam: \\ Camilo Martínez 7057573, cama00005@stud.uni-saarland.de \\ Honglu Ma 7055053, homa00001@stud.uni-saarland.de \\ Maria Sibi 7058469, mama00047@stud.uni-saarland.de}

%
% Title Page
%

\title{
    % \vspace{2in}
    \textmd{\textbf{\hmwkClass:\ \hmwkTitle}}\\
}

\author{\hmwkAuthorName}
\date \today

\renewcommand{\part}[1]{\textbf{\large Part \Alph{partCounter}}\stepcounter{partCounter}\\}

%
% Various Helper Commands
%

% Useful for algorithms
\newcommand{\alg}[1]{\textsc{\bfseries \footnotesize #1}}

% For derivatives
\newcommand{\deriv}[1]{\frac{\mathrm{d}}{\mathrm{d}x} (#1)}

% For partial derivatives
\newcommand{\pderiv}[2]{\frac{\partial}{\partial #1} (#2)}

% Integral dx
\newcommand{\dx}{\mathrm{d}x}

% Define argmax and argmin
\DeclareMathOperator*{\argmax}{argmax}
\DeclareMathOperator*{\argmin}{argmin}

% Alias for the Solution section header
\newcommand{\solution}{\textbf{\large Solution}}

% Probability commands: Expectation, Variance, Covariance, Bias
\newcommand{\E}{\mathrm{E}}
\newcommand{\Var}{\mathrm{Var}}
\newcommand{\Cov}{\mathrm{Cov}}
\newcommand{\Bias}{\mathrm{Bias}}

\begin{document}

\maketitle

\begin{homeworkProblem}{Self-snakes}
    \subsection*{(a)}
    \begin{align*}
        \partial_tu &= |\nabla u|\mathrm{div}\bigl(g(|\nabla u|^2)\frac{\nabla u}{|\nabla u|}\bigr)\\
                    &= |\nabla u|\bigl((g(|\nabla u|^2)\frac{u_x}{|\nabla u|})_x + (g(|\nabla u|^2)\frac{u_y}{|\nabla u|})_y\bigr)\\
                    &= |\nabla u|\bigl(g(|\nabla u|^2)(\frac{u_x}{|\nabla u|})_x + \frac{u_x}{|\nabla u|}(g(|\nabla u|^2))_x + g(|\nabla u|^2)(\frac{u_y}{|\nabla u|})_y + \frac{u_y}{|\nabla u|}(g(|\nabla u|^2))_y\bigr)\\
                    &= |\nabla u|\bigl(g(|\nabla u|^2)\mathrm{div}(\frac{\nabla u}{|\nabla u|}) + \frac{u_x}{|\nabla u|}(g(|\nabla u|^2))_x + \frac{u_y}{|\nabla u|}(g(|\nabla u|^2))_y\bigr)\\
                    &= |\nabla u|\bigl(g(|\nabla u|^2)\mathrm{div}(\frac{\nabla u}{|\nabla u|}) + \frac{u_x\partial_x g(|\nabla u|^2) + u_y\partial_y g(|\nabla u|^2)}{|\nabla u|}\bigr)\\
                    &= |\nabla u|\bigl(g(|\nabla u|^2)\mathrm{div}(\frac{\nabla u}{|\nabla u|})\bigr) + u_x\partial_x g(|\nabla u|^2) + u_y\partial_y g(|\nabla u|^2)\\
                    &= g(|\nabla u|^2)|\nabla u|\mathrm{div}(\frac{\nabla u}{|\nabla u|}) + u_x\partial_x g(|\nabla u|^2) + u_y\partial_y g(|\nabla u|^2)\\
                    &= g(|\nabla u|^2)u_{\xi\xi} + u_x\partial_x g(|\nabla u|^2) + u_y\partial_y g(|\nabla u|^2)\\
                    &= g(|\nabla u|^2)u_{\xi\xi} + u_x \cdot g'(|\nabla u|^2)(2u_xu_{xx} + 2u_yu_{yx}) + u_y \cdot g'(|\nabla u|^2)(2u_xu_{xy} + 2u_yu_{yy})\\
                    &= g(|\nabla u|^2)u_{\xi\xi} + 2g'(|\nabla u|^2)(u_x^2u_{xx} + u_xu_yu_{yx}) + 2g'(|\nabla u|^2)(u_yu_xu_{xy} + u_y^2u_{yy})\\
                    &= g(|\nabla u|^2)u_{\xi\xi} + 2g'(|\nabla u|^2)(u_x^2u_{xx} + 2u_xu_yu_{xy} + u_y^2u_{yy})\\
                    &= g(|\nabla u|^2)u_{\xi\xi} + 2g'(|\nabla u|^2)\nabla u^\top\mathrm{Hess}(u)\nabla u\\
                    &= g(|\nabla u|^2)u_{\xi\xi} + 2g'(|\nabla u|^2)|\nabla u|^2\frac{\nabla u^\top}{|\nabla u|}\mathrm{Hess}(u)\frac{\nabla u}{|\nabla u|}\\
                    &= g(|\nabla u|^2)u_{\xi\xi} + 2g'(|\nabla u|^2)|\nabla u|^2\eta^\top\mathrm{Hess}(u)\eta\\
                    &= g(|\nabla u|^2)u_{\xi\xi} + 2g'(|\nabla u|^2)|\nabla u|^2u_{\eta\eta}\\
    \end{align*}
    \subsection*{(b)}
    The sign in front of $u_{\xi\xi}$ is positive because function $g$ is always positive and
    the sign in front of $u_{\eta\eta}$ is negative because function $g$ is decreasing thus $g'$ is negative.

    The first term is MCM slowed down with a factor of $g$, it is of forward parabolic type.

    The basic property still depends on the contrast parameter $\lambda$ in $g$ because it still exists in $g'$.
    
    We can rewrite the last term in (a) as:
    \begin{align*}
        &2g'(|\nabla u|^2)|\nabla u|^2u_{\eta\eta}\\
       =&\underbrace{2g'(|\nabla u|^2)|\nabla u||u_{\eta\eta}|}_{\le 0}\mathrm{sgn}(u_{\eta\eta})|\nabla u|\\
    \end{align*}
    which resembles a shock filter.
\end{homeworkProblem}
\begin{homeworkProblem}{Minmod Scheme for SILD}
    \subsection*{(a)}
    \[
        \frac{du_i}{dt} = -\frac{1}{h} (\mathrm{MM}(u^x_{i+1}, u^x_i, u^x_{i-1}) - \mathrm{MM}(u^x_{i}, u^x_{i-1}, u^x_{i-2}))
    \]
    \subsection*{(b)}
    It will make the derivative 0 at signal extrema (i.e. no change for $u$).

    \subsection*{(c)}
    Monotonically increasing concave? Isn't that always increasing and there is no extrema?
    Why do we still need Minmod?
\end{homeworkProblem}

\end{document}
